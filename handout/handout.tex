\documentclass[letterpaper, 12pt, titlepage, twoside]{article}
\input setup

\title{General-Purpose Programming in Python}
\author{P. K. G. Williams (\href{mailto:peter@newton.cx}{peter@newton.cx})}
\date{\today \\ version \t{\gitversion}}

\begin{document}
\maketitle

\section*{Introduction\markboth{Introduction}{}}

This handout will walk you through some of the key concepts of general-purpose
computer programming using the Python language.

We assume that you are using an interactive Jupyter notebook to execute Python
commands in small pieces. Code that you should type is displayed this way:

\begin{lstlisting}
  print('Hello, world!')
  x = 1
\end{lstlisting}

When using Jupyter, you should end most lines by just hitting the \s{Enter}
key as usual. To actually \i{run} the block of code that you have just typed,
finish the final line by hitting \s{Shift-Enter} (that is, the \s{Enter} key
with the \s{Shift} key also held down).

We are \i{very intentionally} forcing you to type the code yourself! We feel
that this is very important for learning how to program in any particular
language. When you're writing your own programs, you'll certainly have to type
everything yourself, and there's no time like the present to start practicing.

As ever, please be very careful about typing in commands \i{exactly} as they
appear in printed form. Punctuation characters that look quite similar are
often interpreted quite differently by Python.

Jupyter has a smart editor that tries to automatically manage the
\i{indentation} of your code --- the white space at the beginning of each
line. Its automatic indentation should line up with what we show here. If it
doesn't, please ask a helper for advice. Using an editor with proper automatic
indentation is \i{enormously} important when coding in Python.


\newpage
\section{Foundations}

TBD.


\newpage
\section{Data structures}

Yo yo.

\begin{lstlisting}
  x = 1
  y = 2 * x
\end{lstlisting}

Carry on.

\end{document}
