\documentclass[letterpaper, 12pt, titlepage, twoside]{article}
\input setup

\title{General-Purpose Programming in Python}
\author{P. K. G. Williams (\href{mailto:peter@newton.cx}{peter@newton.cx})}
\date{\today \\ version \t{\gitversion}}

\begin{document}
\maketitle

\section*{Introduction\markboth{Introduction}{}}

This handout will walk you through some of the key concepts of general-purpose
computer programming using the Python language.

We assume that you are using an interactive Jupyter notebook to execute Python
commands in small pieces. Code that you should type is displayed this way:

\begin{lstlisting}
  print('Hello, world!')
  x = 1
\end{lstlisting}

When using Jupyter, you should end most lines by just hitting the \s{Enter}
key as usual. To actually \i{run} the block of code that you have just typed,
finish the final line by hitting \s{Shift-Enter} (that is, the \s{Enter} key
with the \s{Shift} key also held down). Each numbered chunk of code that you
type in your Jupyter notebook is called a \i{cell}. \i{You must separate cells
  in the notebook in exactly the way written here in order to get the desired
  outputs}.

We are \i{very intentionally} forcing you to type the code yourself! We feel
that this is very important for learning how to program in any particular
language. When you're writing your own programs, you'll certainly have to type
everything yourself, and there's no time like the present to start practicing.

As ever, please be very careful about typing in commands \i{exactly} as they
appear in printed form. Punctuation characters that look quite similar are
often interpreted quite differently by Python.

One exception: we'll write some explanatory text as Python comments, which are
delimited with hash signs (\s{\#}). You can skip typing the \s{\#} and
everthing that comes after it.

\begin{lstlisting}
  x = 7 # now x is 7, but you don't need to type this part
\end{lstlisting}

Every so often we intentionally have you type a command that results in an
error. We will indicate this with a comment about ``raising'' a particular
kind of error.

\begin{lstlisting}
  assert False # !!! raises AssertionError
\end{lstlisting}

\i{Only} lines with these annotations should result in error messages. These
lines will come at the ends of cells, since any code within a cell that comes
after an error will not be run. Every statement \i{before} the problematic
line, however, will have taken effect.

Jupyter has a smart editor that tries to automatically manage the
\i{indentation} of your code --- the white space at the beginning of each
line. Its automatic indentation should line up with what we show here. If it
doesn't, please ask a helper for advice. Using an editor with proper automatic
indentation is \i{enormously} important when coding in Python.

As you might expect, the topic of ``general-purpose programming'' is very
broad. Meanwhile there is only a small amount of time available for this
lesson. Given these constraints we have opted to go over a relatively large
number of topics while spending only a small amount of time on each on. So,
unfortunately, \i{this lesson will be hurried, and many ideas will be
  presented only superficially}. Sorry, there's just too much to learn!


% Haven't come up with a hands-on portion that seems worthwhile.
%%%\newpage
%%%\section{Foundations}


\newpage
\section{Data structures}

In this activity we will practice creating various kinds of Python data
objects. Our exploration will be \i{very} superficial.

\subsection*{Foundational data types}

The five foundational data types in Python are \i{the None type} (representing
an empty or undefined value), \i{booleans} (true/false values), \i{integers},
\i{floats} (approximations of real numbers), and \i{strings} (sequences of
textual characters).

Here are some examples of ways to assign variables to these different kinds of
values. Please type all of these in.

\begin{lstlisting}
  a_none_variable = None
  a_boolean = False
  other_boolean = True
  positive_int = 37
  negative_int = -100
  positive_float = 1.75
  one_million = 1e6
  negative_pi = -3.14159
  a_string = "Hello."
  other_string = 'single quotes work too'
  empty_string = ""
  example_complex_string = '''\x46a\x6E\x63y\tth\x69n\x67s'''
\end{lstlisting}

\i{When using the Jupyter notebook}, if you just type in a variable's name in
a cell and hit \s{Shift-Enter}, you will be shown the current value of that
variable:

\begin{lstlisting}
  one_million
\end{lstlisting}

If you want to look at multiple variables, you have to put each one's name in
a different cell:

\begin{lstlisting}
  example_complex_string
\end{lstlisting}

You can do math with these basic data types using about the notation that you
would expect. Multiplication is expressed with a single asterisk.
Exponentiation ($a^b$) is written with a double asterisk. You can ``add'' two
string variables, which results in their \i{concatenation}: the first string
followed by the second string.

\begin{lstlisting}
  boolean_logic = a_boolean and (not other_boolean)
  new_number = positive_int * negative_int
  other_number = (positive_int / positive_float)**2
  concatted_strs = 'Hello,' + ' world.'
\end{lstlisting}

\subsection*{Core data structures}

Python has three basic data types for building data structures. They are
generally referred to as \i{collections} because they contain, well,
collections of other data.

\i{Lists} are
ordered sequences of other data. Lists can hold any kind of data.

\begin{lstlisting}
  a_list_of_ints = [1, 2, 5]
  a_mixed_list = ['Hello', -3.14159, one_million, a_list_of_ints]
\end{lstlisting}

You can access the \i{items} (also called \i{elements}) of a list by
\i{indexing} with square brackets.

\medskip
\begin{center}
  \fbox{\Large The first list index is zero, not one!}
\end{center}
\medskip

Every single person who starts out programming has trouble getting used to
this. Rest assured: there are very good reasons for this convention that will
become clear over time.

\begin{lstlisting}
  a_list_of_ints[0]
\end{lstlisting}

\begin{lstlisting}
  a_list_of_ints[2]
\end{lstlisting}

Here are some terse examples hinting at what you can do with lists.

\begin{lstlisting}
  length_of_my_list = len(a_list_of_ints)
  a_list_of_ints.append(8)
  print(len(a_list_of_ints))
  print(a_list_of_ints[-1])
  print(a_mixed_list + a_list_of_ints)
\end{lstlisting}

Python's \i{sets} are like mathematical sets: unordered collections of
objects, without duplication. Because of this latter constraint, you can only
construct sets of certain kinds of data; mutable data structures are not
allowed, because you could mutate them to break the constraint. Sets \i{can}
combine different types of data, but tricky subtleties can arise.

Because sets are unordered, you cannot index them. If you want to get a set's
items back out, often it is helpful to convert it into a list with the
\t{list()} function --- keeping in mind that \i{the list you get back will be
  given to you in an unpredictable order}.

\begin{lstlisting}
  a_set = {1, 5, 10, 15}
  other_set = {1, 1.74, 'hello'}
  empty_set = set() # note: "{}" doesn't work
  list_converted_to_set = set(a_list_of_ints)
\end{lstlisting}

Here are some things you can do with sets:

\begin{lstlisting}
  print(len(a_set))
  print(len(empty_set))
  print(1 in a_set)
  print(17 in a_set)
  print(17 not in a_set)
  a_set.add(17)
  print(17 not in a_set)
  print(list(a_set))
\end{lstlisting}

Finally, \i{dictionaries} (usually shortened to \i{dicts}) are unordered
collections of pairs of unique \i{keys} associated with possibly-non-unique
\i{values}. Given a key, it is easy to fetch a value, but the opposite is not
true. Relatedly, because dictionary keys must be unique, they have to follow
the same constraints as set items. Dictionary values do not have this
constraint.

\begin{lstlisting}
  a_dict = {'key1': 'value1', 'key2': 'value2'}
  other_dict = {1: [1, 2, 3], -37: None}
  dict_with_string_keys = dict(key1='value1', key2='value2')
  empty_dict = {} # this syntax always gives you a dict, not a set
\end{lstlisting}

To look up a value in a dict, you index it, using the same typed syntax as you
do for a list. You can use variable-assignment syntax to add new entries to
dicts. You can use ``\s{del} syntax'' to remove entries.

\begin{lstlisting}
  print(a_dict['key1'])
  print(len(a_dict))
  a_dict['newkey'] = 'newvalue'
  print(len(a_dict))
  del a_dict['key1']
  print(len(a_dict))
  print(list(other_dict.keys()))
  print(list(other_dict.values()))
\end{lstlisting}

The three basic structures provided by Python are \i{not} the only ones out
there, and they are not necessarily ``fundamental'' in any deep sense. For
instance, lists can be thought of as dicts where the keys are integers
starting at zero. Likewise, sets can be thought of as dicts where the values
are all \s{True}.

\subsection*{Mutability in action}

In the mini-lecture, we emphasized the importance of \i{mutability} and
\i{ownership} when thinking about data. The basic data structures are mutable,
so they can change from under you in surprising ways. Try hard to understand
what happens when you run this small example:

\begin{lstlisting}
  my_list = [1, 2, 3]
  print(my_list)
  other_list = my_list # "other_list" is a new name for the same data
  other_list.append(37)
  print(my_list)
\end{lstlisting}

The foundational data types are all immutable, so they are not susceptible to
these kinds of surprises.

Python does, however, provide versions of its data structures that are
immutable after creation. The immutable version of a list is called a
\i{tuple}. It is usually written with parentheses. However, in many cases the
parentheses are actually optional!

\begin{lstlisting}
  a_tuple = (0, 3, 4)
  equivalent_tuple = 0, 3, 4
  other_tuple = (a_dict, 'hello', empty_set)
  empty_tuple = ()
\end{lstlisting}

In Python, the syntax \s{(x)} is interpreted as mathematical grouping. To
create a tuple with one element, you must use a special syntax:

\begin{lstlisting}
  one_item_tuple = (True,)
\end{lstlisting}

Because of mutability, lists cannot be used as dictionary keys, but tuples can:

\begin{lstlisting}
  my_dict = {}
  my_dict[(1,2,3)] = 'hello'
  my_dict[1,2,4] = 'goodbye' # tuple parentheses can be omitted here
  print(my_dict)
  my_dict[[1,2,5]] = 'problem' # !!! raises TypeError
\end{lstlisting}

You can generally access and index tuples as you would lists. While you cannot
modify tuples' contents directly, tuples \i{can} contain data structures that
are themselves mutable:

\begin{lstlisting}
  print(other_tuple)
  a_dict['morekey'] = 'edited'
  print(other_tuple)
\end{lstlisting}

Because of this fact, tuples containing mutable data structures
\i{cannot} be used as keys for dictionaries or sets.

\begin{lstlisting}
  my_dict[other_tuple] = 'problem' # !!! raises TypeError
\end{lstlisting}

The immutable version of a set is called a \s{frozenset}. There is no
built-in syntax to create frozen sets. You must instead create them with the
\s{frozenset()} function. This function takes one argument that is a
collection, so be careful with parentheses.

\begin{lstlisting}
  frozen_from_list = frozenset([1, 2, 3])
  some_tuple = (4, 5, 6)
  frozen_from_tuple = frozenset(some_tuple)
  frozen_from_equivalent_tuple = frozenset((4, 5, 6))
  too_many_arguments = frozenset(4, 5, 6) # !!! raises TypeError
\end{lstlisting}

There is no built-in immutable version of a \s{dict}.

\subsection*{Choosing the Right Data Structure}

In realistic programs it is \i{exceedingly} important to choose the right data
structure for your job. Fortunately, Python's foundational structures are very
flexible and can gracefully handle all sorts of tasks. But there are more
choices to be made than you might initially think.

For example, say that your program works with radio measurements. Each
measurement (\s{m}) is tagged with a time (\s{t}), a frequency (\s{f}), and a
polarization (\s{p}). Here are some ways that you could store a collection of
measurements:

\begin{itemize}
\item A list of tuples, each tuple of the form \s{(t, f, p, m)}. Each
  polarization is encoded as a string: \s{"nn"}, \s{"ee"}, \s{"ne"}, or
  \s{"en"}.
\item The same as the above, except the polarizations are encoded as the
  integers 0--3 instead of strings.
\item A dict of lists, where the dict keys are the four polarizations
  (\s{"nn"}, etc.), and the lists are made of tuples \s{(t, f, m)}.
\item The same as the above, but the dict keys are the timestamps
  (\s{t}), and the lists are made of tuples \s{(f, p, m)}.
\item A tuple of four lists: one of all the timestamps, one of frequencies,
  one of polarizations, and one of measurements.
\item A dict of dicts of dicts of floats, where the outermost keys are times,
  then frequencies, then polarizations, and the final values are the
  measurements.
\end{itemize}

Which of these is the best? It depends on the characteristics of your data and
what task you're trying to accomplish. When deciding how to represent your data in
Python structures, think about these factors:

\begin{enumerate}
\item Which choice will help me write the simplest, clearest code?
\item Which choice will result in the fastest program?
\item Which choice will result in the most memory-efficient program?
\end{enumerate}

Generally, the first consideration is the most important and the last is the
least important. When you're dealing with large data sets, however, you may
have to trade off code complexity for efficiency.

Very similar considerations apply when writing data to disk and reading them
again. The file format that you use to save data also involves tradeoffs for
speed, efficiency, convenience, and so on. It is \i{very common} to see people
spend a lot of time wrestling with code that reads and writes data because the
file format they are using is not well-matched to the problem they are trying
to solve. If you're having trouble doing what you want with a data set, don't
forget to ask yourself: \i{is the problem in the code, or in the data
  structure}?


\newpage
\section{Input and Output (``IO'')}

We've already encountered one form of IO: the \s{print()} function. It turns
your Python data into strings and prints them to your Jupyter notebook or, if
you're writing a command-line program, to your terminal.

\subsection*{We are Very, Very Sorry}

We now have to spend some time on boring technical history. You shouldn't need
to worry about it, but as things stand in the world of Python, you do.

We have told you that there are are two main versions of Python out there,
Python~2 and Python~3, and that we are using Python~2. One of the key
differences between the two is that in Python~2, \s{print} is actually a
special form of syntax in the language, not a built-in function. This was
eventually recognized to be an enormous annoyance, so it was fixed in
Python~3.

In Python~3, you can print multiple values in one line by giving the print
function multiple arguments. The values will be separated by spaces in the
output. The resulting code might look like:

\begin{lstlisting}
  x = 17
  print('The value of x is', x) # Python 3 style, but won't crash Python 2
\end{lstlisting}

In Python~2, there are no parentheses:

\begin{lstlisting}
  print 'The value of x is', x # Python 2 style; will fail in Python 3
\end{lstlisting}

In the examples in this handout thus far, we have been tricky. Because
parentheses around a single value are interpreted by Python to represent
grouping, you can get consistent results in both versions of Python when
printing single values:

\begin{lstlisting}
  print(x) # same result in both Python 2 and Python 3
\end{lstlisting}

This is the reason that the first example, with parentheses, doesn't crash in
Python~2, although it gives results that are probably not what you want.
Version~2 of Python interprets the first example as asking it to print out a
tuple made out of a string and an integer. This is close to what you want, but
it adds parentheses and commas around the values that get printed.

\begin{lstlisting}
  python_2s_view = ('The value of x is', x)
  print python_2s_view
\end{lstlisting}

If you are writing a Python script, you can add a special command that will
give you ``print as a function'' behavior in \i{both} Python~3 and Python~2.
At the top of your file, you can write:

\begin{lstlisting}
  from __future__ import print_function # !!! might raise error in Jupyter
\end{lstlisting}

This will activate a special mode in Python~2 and do nothing in Python~3. You
should use this syntax because it helps your code be compatible with Python~3
and the ``print as a function'' mode is more sensible and flexible overall.

\subsection*{Back to the IO}

The built-in \s{open()} function opens regular files on your computer's hard
disk for input or output. We refer to the file's \i{open mode} as either being
\i{read} or \i{write} depending on what you're going to do with it. The mode
is denoted with a special string argument that is passed to \s{open()}.

\begin{lstlisting}
  f = open('mydata.txt', 'w') # the "w" means to open for writing
\end{lstlisting}

Here's an important thing to know: \i{when you open a file for writing, the
  operating system will delete all of its contents and give you an empty
  file!} There is a special \s{"a"} mode for \i{appending} data to an existing
file without erasing its contents. It may seem surprising at first, but often
the default behavior (\i{truncation}) is preferable to appending.

The \s{open()} function returns a new form of Python data, an \i{object} --- a
\i{file object}, to be specific. You can assign a file object to a variable
like any other kind of Python data, but it is not a string, or a dict, or one
of the other basic data structure types. We say that it is \i{opaque}: it's
like a box that you can hold and move around, but you can't see exactly what's
inside it.

File objects, like almost all other Python objects, come with functions
attached to them. You can access and call these functions using ``dot syntax''
as shown below. Here the \s{'\textbackslash n'} bit of the string is a special
``escape code'' that indicates that a new line of output should start.

\begin{lstlisting}
  f.write('Here is some text.\n')
  another_name_for_f = f
  another_name_for_f.write('Here is more text.\n')
\end{lstlisting}

There is a special way to invoke the \s{print} statement that prints data to a
file instead of to the terminal:

\begin{lstlisting}
  print >>f, 'The value of x is', x
\end{lstlisting}

When you're done with a file that you've opened, you should always close it:

\begin{lstlisting}
  f.close()
\end{lstlisting}

Once you've closed a file, the \s{f} variable is in a bit of a funny state:
the variable still exists, but it's illegal to do anything with it:

\begin{lstlisting}
  print >>f, 'This will fail' # !!! raises ValueError
\end{lstlisting}

One reason that it's important to close files is \i{buffering}: for efficiency
reason, Python will not actually write data to disk every time you issue a
\s{print} or \s{write()} command. The last data in your file are very unlikely
to actually show up until you issue the \s{close()} call. While Python will
eventually write out your data if you forget the \s{close()}, including it
gives you precise control over when this happens.

The above is just one example of the subtleties that arise when doing IO.
There are \i{many} more. What happens when two programs try to write to the
same file? What happens when the file is accessed over a network and the
network goes down? What happens if a file is deleted while you're reading it?
The answers to these questions are defined by the underlying operating system,
not Python. Since every useful program has to do IO, it is \i{very} worthwhile
to spend time learning about the fundamental rules of IO on Unix-like
operating systems.

\subsection*{Network IO}

Thanks to the infrastructure that Python provides, reading and writing over
the network is almost as easy as reading and writing files from the local hard
disk. For instance, downloading a file from the internet takes just a few
lines of code:

\begin{lstlisting}
  from urllib2 import urlopen
  from shutil import copyfileobj
  source = urlopen('http://tinyurl.com/l66kod5')
  dest = open('datatable.txt', 'w')
  copyfileobj(source, dest)
  source.close()
  dest.close()
\end{lstlisting}

Above, the first two lines load standard Python \i{modules} that provide
pre-packaged routines that allow us to avoid a lot of work that we'd otherwise
have to do ourselves.

\subsection*{High-Level IO}

The \s{print} statement is useful but it is not a good solution for anything
other than ad-hoc diagnostics and messages intended for users.

For more important data, it is better to choose a preexisting, standard data
format that is suited to the data. No matter what format you choose, there
will probably be a Python \i{module} that provides functions that make it easy
to read and write the data format.

For instance, the file we downloaded contains data in the textual table format
developed by the Centre de Données astronomiques de Strasbourg (CDS). The
popular Python module \href{http://www.astropy.org/}{AstroPy} can read these
files into customized table data objects.

\begin{lstlisting}
  from astropy.io import ascii
  table = ascii.read('datatable.txt')
  table
\end{lstlisting}

Here, the AstroPy table object has special integration with Jupyter so that
when we display its value in the notebook, you get a fancy representation of
the table contents.


\end{document}
